% Copyright (c) 2005-2009 Nokia Corporation
%
% Licensed under the Apache License, Version 2.0 (the "License");
% you may not use this file except in compliance with the License.
% You may obtain a copy of the License at
%
%     http://www.apache.org/licenses/LICENSE-2.0
%
% Unless required by applicable law or agreed to in writing, software
% distributed under the License is distributed on an "AS IS" BASIS,
% WITHOUT WARRANTIES OR CONDITIONS OF ANY KIND, either express or implied.
% See the License for the specific language governing permissions and
% limitations under the License.

\section{\module{contacts} ---
  A contacts related services package}
\label{sec:contacts}

\declaremodule{extension}{contacts}
\platform{S60}
\modulesynopsis{A contacts related services package.}

The \module{contacts} module offers an API to address book services allowing the 
creation of contact information databases. The \module{contacts} module 
represents a Symbian contact database as a dictionary-like \class{ContactDb} 
object, which contains \class{Contact} objects and which is indexed using the 
unique IDs of those objects. A \class{Contact} object is itself a list-like 
object, which  contains \class{ContactField} objects and which is indexed using 
the field indices. Unique IDs and field indices are integers. The 
\class{ContactDb} object supports a limited subset of dictionary functionality. 
Therefore, only \code{__iter__}, \code{__getitem__}, 
\code{__delitem__},\code{__len__}, \code{keys}, \code{values}, and \code{items} 
are included.

\class{ContactDb} objects represent a live view into the database. If a 
contact is changed outside your Python application, the changes are visible 
immediately, and conversely any changes you commit into the database are 
visible immediately to other applications. It is possible to lock a contact 
for editing, which will prevent other applications from modifying the 
contact for as long as the lock is held. This can be done in, for example, a 
contacts editor application when a contact is opened for editing, very much 
like with the Contacts application in your Nokia device. If you try to 
modify a contact without locking it for editing, the contact is 
automatically locked before the modification and released immediately 
afterwards.

\subsection{Module Level Functions}
\label{subsec:contmod}

The following free functions - functions that do not belong to any class 
- are defined in the \class{Contact} module:

\begin{funcdesc}{open}{\optional{filename\optional{, mode}}}

Opens a contacts database and returns a \class{ContactDb} object. \var{filename} 
should be a full Unicode path name. If \var{filename} is not given, opens the 
default contacts database. If \var{mode} is not given, the database must exist. 
If \var{mode} is '\code{c}', the database is created if it does not already 
exist. If \var{mode} is '\code{n}', a new, empty database is created, 
overwriting the possible previous database.

\end{funcdesc}

\begin{notice}[warning]
Using \code{open} together with the additional parameters \var{filename} 
or \var{mode} is intended for testing purposes only. Due to S60 SDK 
functionality, the \code{open} method can sometimes be unreliable with these 
parameters. 
\end{notice}

\subsection{ContactDb Object}
\label{subsec:contactdb}

There is one default contact database, but it is possible to create several 
databases with the \code{open} function.

\begin{classdesc*}{ContactDb}
\class{ContactDb} objects have the following methods:

\begin{methoddesc}[ContactDb]{add_contact}{}

Adds a new contact into the database. Returns a \class{Contact} object that 
represents the new contact. The returned object is already locked for 
modification. Note that a newly created contact will contain some empty 
default fields. If you do not want to use the default fields for anything, 
you can ignore them.
 
\end{methoddesc}

\begin{methoddesc}[ContactDb]{find}{searchterm}

Finds the contacts that contain the given Unicode string as a substring and 
returns them as a list.

\end{methoddesc}

\begin{methoddesc}[ContactDb]{import_vcards}{vcards}

Imports the vCard(s) in the given string into the database. 

\end{methoddesc}

\begin{methoddesc}[ContactDb]{export_vcards}{ids}

Converts the contacts corresponding to the ID's in the given tuple \var{ids} to vCards and returns 
them as a string.

\end{methoddesc}

\begin{methoddesc}[ContactDb]{keys}{}

Returns a list of unique IDs of all \code{Contact} objects in the 
database.

\end{methoddesc}

\begin{methoddesc}[ContactDb]{compact_required}{}

Verifies whether compacting is recommended. Returns an integer value 
indicating either a true or false state. Returns \code{True} if more than 
32K of space is unused and if this comprises more than 50 percent of the 
database file, or if more than 256K is wasted in the database file. 

\end{methoddesc}

\begin{methoddesc}[ContactDb]{compact}{}

Compacts the database to its minimum size.

\end{methoddesc}

\begin{methoddesc}[ContactDb]{__delitem__}{id}

Deletes the given contact from the database.

\end{methoddesc}


\begin{methoddesc}[ContactDb]{field_types}{}

Returns a list of dictionary objects that contains information on all 
supported field types. The list contains dictionary objects, which each 
describe one field type. The most important keys in the dictionary are 
\code{'type'} and \code{'location'} which together indentify the field 
type. \code{'type'} can have string values such as 
\code{'email_address'}. \code{'location'} can have the string values 
\code{'none'}, \code{'home',} or \code{'work'}. Another important key 
is \code{'storagetype'}, which defines the storage type of the field. 
\code{'storagetype'} can have the string values \code{'text'}, 
\code{'datetime'}, \code{'item_id',} or \code{'binary'}. Note that 
the \code{Contacts} extension does not support adding, reading, or 
modifying fields of any other type than '\code{text'} or 
\code{'datetime'}. The other content returned by \code{field_types} 
is considered to be advanced knowledge and is not documented here.

\end{methoddesc}

\begin{memberdesc}[ContactDb]{groups}

Returns contact groups of the database. Read-only.

\end{memberdesc}

\end{classdesc*}

\subsection{Contact Object}
\label{subsec:contact}

A \class{Contact} object represents a live view into the state of a single 
contact in the database. You can access the fields either with a contact's 
numeric field ID as \code{contact[fieldid]}, or using the \code{find} 
method. Attempting to modify a contact while it has been locked for editing 
in another application will raise the exception \code{ContactBusy}.

\begin{classdesc*}{Contact}
\class{Contact} objects have the following attributes:

\begin{memberdesc}[Contact]{id}

The unique ID of this \class{Contact}. Read-only.

\end{memberdesc}

\begin{memberdesc}[Contact]{title}

The title of this \class{Contact}. Read-only.

\end{memberdesc}

\begin{memberdesc}[Contact]{last_modified}

The date/time when this \class{Contact} object was last modified. Read-only.

\end{memberdesc}

\begin{memberdesc}[Contact]{is_group}

Returns 1 if this contact is a contact group. Returns 0 if normal contact entry. Read-only.

\end{memberdesc}

\class{Contact} objects have the following methods:

\begin{methoddesc}[Contact]{begin}{}

Locks the contact for editing. This prevents other applications from 
modifying the contact for as long as the lock is held. This method will 
raise the exception \code{ContactBusy} if the contact has already been 
locked.

\end{methoddesc}

\begin{methoddesc}[Contact]{commit}{}

Releases the lock and commits the changes made into the database.

\end{methoddesc}

\begin{methoddesc}[Contact]{rollback}{}

Releases the lock and discards all changes that were made. The contact 
remains in the state it was before \code{begin}.

\end{methoddesc}

\begin{methoddesc}[Contact]{as_vcard}{}

Returns the contact as a string in vCard format.

\end{methoddesc}

\begin{methoddesc}[Contact]{add_field}{type \optional{, value \optional{, 
label=field_label}\optional{, location=location_spec}}}

Adds a new field into this \class{Contact}. This method raises 
\code{ContactBusy} if the contact has been locked by some other 
application. \var{type} can be one of the supported field types as a 
string. 

The following field types can be added:

\begin{itemize}
\item \code{city}
\item \code{company_name}
\item \code{country}
\item \code{date}
\item \code{dtmf_string}
\item \code{email_address}
\item \code{extended_address}
\item \code{fax_number}
\item \code{first_name}
\item \code{job_title}
\item \code{last_name}
\item \code{mobile_number}
\item \code{note}
\item \code{pager_number}
\item \code{phone_number}
\item \code{po_box}
\item \code{postal_address}
\item \code{postal_code}
\item \code{state}
\item \code{street_address}
\item \code{url}
\item \code{video{\_}number}
\item \code{picture}
\item \code{second_name}
\item \code{voip}
\item \code{sip_id}
\item \code{personal_ringtone}
\item \code{share_view}
\item \code{prefix}
\item \code{suffix}
\item \code{push_to_talk}
\item \code{locationid_indication}


\end{itemize}

The following field types are recognized but cannot be created at present:

\begin{itemize}
\item \code{first{\_}name{\_}reading}
\item \code{last{\_}name{\_}reading}
\item \code{speed_dial}
\item \code{thumbnail{\_}image}
\item \code{voice_tag}
\item \code{wvid}
\end{itemize}



All supported field types are passed as strings or Unicode strings, except for 
\code{'date}' which is a float that represents Unix time. For more information 
on Unix time, see Section \ref{subsec:datetime}, Date and Time.

\var{field{\_}label} is the name of the field shown to the user. If you 
do not pass a label, the default label for the field type is used.

\var{location{\_}spec}, if given, must be \code{'home'} or 
\code{'work'}. Note that not all combinations of type and location are 
valid. The settings of the current contacts database in use determine which 
ones are valid. 

\end{methoddesc}

\begin{methoddesc}[Contact]{find}{\optional{type=field{\_}type}\optional{, 
location=field{\_}location}}

Finds the fields of this contact that match the given search specifications. 
If no parameters are given, all fields are returned.

\end{methoddesc}

\begin{methoddesc}[Contact]{__delitem__}{fieldindex}

Deletes the given field from this contact. Note that since this will change 
the indices of all fields that appear after this field in the contact, and 
since the \class{ContactField} objects refer to the fields by index, old 
\class{ContactField }objects that refer to fields after the deleted field 
will refer to different fields after this operation.

\end{methoddesc}

\end{classdesc*}

\subsection{ContactField Object}
\label{subsec:contactfield}

A \class{ContactField} represents a field of a \class{Contact} at a 
certain index. A \class{ContactField} has attributes, some of which can be 
modified. If the parent \class{Contact} has not been locked for editing, 
modifications are committed immediately to the database. If the parent 
\class{Contact} has been locked, the changes are committed only when 
\class{commit} is called on the \class{Contact}.

\begin{classdesc*}{ContactField}
\class{ContactField} objects have the following attributes:

\begin{memberdesc}[ContactField]{label}

The user-visible label of this field. Read-write.

\end{memberdesc}

\begin{memberdesc}[ContactField]{value}

The value of this field. Read-write.

\end{memberdesc}

\begin{memberdesc}[ContactField]{type}

The type of this field. Read-only.

\end{memberdesc}

\begin{memberdesc}[ContactField]{location}

The location of this field. This can be \code{'none'}, \code{'work'}, or 
\code{'home'}.

\end{memberdesc}

\begin{memberdesc}[ContactField]{schema}

A dictionary that contains some properties of this field. The contents of this 
dictionary correspond to those returned by the \class{ContactDb} method 
\method{field{\_}types}.

\end{memberdesc}

\end{classdesc*}



\subsection{Groups Object}
\label{subsec:groups}

A \class{Groups} object represents Symbian contact groups as a dictionary
like object with limited subset of dictionary functionality. Each group can
be accessed using the group's unique id as a key. The Groups object returns 
a list like \class{Group} object as the value matching the given key.

The following common methods are supported: \code{__iter__}, \code{__getitem__},
\code{__delitem__} and \code{__len__}.

\begin{classdesc*}{Groups}
\class{Groups} objects have the following attributes:

\begin{methoddesc}[Groups]{add_group}{\optional{name}}

Creates new contact group and returns corresponding \class{Group} object.
Group name can be given as an optional parameter.

\end{methoddesc}

\end{classdesc*}



\subsection{Group Object}
\label{subsec:group}

A \class{Group} object represents single Symbian contact group as a list object 
with limited subset of list functionality. The \class{Group} object lists Contact
entry ids that belong to the group.

The native Symbian group objects are represented as Symbian contact entries in
the database. Therefore they can also be accessed as Python \class{Contact} objects,
 but this way their group handling properties cannot be used from Python. Use 
\class{Groups} and \class{Group} objects to access group functionalities. 

The following common methods are supported: \code{__iter__}, \code{__getitem__},
\code{__delitem__} and \code{__len__}.

\begin{classdesc*}{Group}
\class{Group} objects have the following attributes:

\begin{memberdesc}[Group]{id}

The unique id of the \class{Group} object. Read-only.

\end{memberdesc}

\begin{memberdesc}[Group]{name}

The name of the \class{Group} object. Read-write.

\end{memberdesc}

\end{classdesc*}

\newpage










